We consider an undirected weighted graph $G = (V, E, \W)$ without self-loops, where $V = \set{v_1, \dots, v_{\card{V}}}$ is the set of $\card{V}$ vertices, $E = \set{e_{i,j} = \parentheses{v_i, v_j}}$ is the set of $\card{E}$ edges defined for some $i < j$, and $\W \in \euclidean{\card{E} \times \card{E}}$ is a diagonal matrix collecting weights $w_{i,j} > 0$ associated with $e_{i, j}$.
The $G$ can be characterized by its (unnormalized) graph Laplacian $\L = \transpose{\D} \W \D \in \euclidean{|V| \times |V|}$, with the oriented incidence matrix $\D \in \mathbb{R}^{|E| \times |V|}$ defined by
\begin{equation}
    \D_{e_{i,j}, v_k} \coloneqq
    \begin{cases}
        -1, & \text{if } i = k, \\
        1,  & \text{if } j = k, \\
        0,  & \text{otherwise}.
    \end{cases}
\end{equation}

A graph signal is a vector $\x \in \euclidean{\card{V}}$ with $x_i$ defined on a vertex $v_i$. 
Two common measures of signal variation on $G$ are
\begin{align}
    \label{eq:glr}
    \transpose{\x} \L \x &= \sum_{i,j} w_{i,j} (x_i - x_j)^2, \\
    \label{eq:gtv}
    \lone{\W \D \x} &= \sum_{i,j} w_{i,j} \abs{x_i - x_j}.
\end{align}
The first quantity corresponds to \emph{graph Laplacian regularization (GLR)}, which promotes global smoothness of the signal, while the second corresponds to \emph{graph total variation (GTV)}, which promotes piecewise smoothness. 

GLR is highly effective when edge weights accurately reflect signal similarity, but it is sensitive to weight inaccuracies. 
In contrast, GTV is more robust to uncertain edges since large discrepancies across unreliable connections do not dominate the regularization term. 
However, this robustness comes at the expense of under-penalizing small but meaningful differences on reliable edges, limiting its effectiveness compared with GLR.