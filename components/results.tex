\begin{table}[t]
    \small
    \caption{RMSEs averaged over $20$ independent runs, with the best and the second best performance highlighted by bolded and underlined, respectively. Our proposed regularizations consistently achieved better performances over GLR and GTV.}
    \label{visuals:quantitative_results}
    \centering
    \begin{tabular}{c|cccc}
        \toprule
        Task & GLR & GTV & Ours (\ref{equation:proposed_1}) & Ours (\ref{equation:proposed_2}) \\
        \midrule
		Denoising	& $0.1757$ & $0.1640$ & $\underline{0.1629}$ & $\mathbf{0.1485}$ \\
        Inpainting 	& $0.1714$ & $0.1848$ & $\underline{0.1697}$ & $\mathbf{0.1353}$ \\
		+ noise 	& $0.2549$ & $0.2637$ & $\underline{0.2500}$ & $\mathbf{0.2300}$ \\

        \bottomrule
    \end{tabular}
\end{table}
Table \ref{visuals:quantitative_results} shows the RMSEs averaged over $20$ independent runs on each task and regularization, with the best and second best performance highlighted by bolded and underlined, respectively.
Our proposed regularizations surpassed GLR and GTV across all tasks, validating their robust smoothness-promoting ability over GLR against inaccurate graph weights.
In particular, the proposed method (\ref{equation:proposed_2}) consistently achieved the best performance, which implies the importance of incorporating the exact GTV in the error-tolerant term.

Fig. \ref{visuals:qualitative_results} visualizes an example of recovery results on each method in the noiseless inpainting task, where the edges with corrupted weights are highlighted in thicker black.
The first column shows the true signal in the upper and the observed signal in the lower, with the masked signal elements as cross symbols.
In the other columns, the first row shows the recovered signals, and the second row highlights the error between the true and restored signals.
This figure illustrates the assumptions that inspired our regularization design: GLR and GTV failed on different nodes, indicating their complementary advantages against weight errors.
Our proposed regularizations successfully inherited these strengths, as they accurately restored the signal on nodes where GLR and GTV struggled.