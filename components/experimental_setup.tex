We experimented with a sensor graph with $64$ vertices and $236$ edges provided in GSPBox \cite{perraudin2014gspbox}, where the weights were synthesized to construct the true and corrupted graphs, then scaled so that the mean was $1$.
The true weights were generated by deflating uniform weights by $0.1$ at randomly selected $50\%$ of edges, and the corrupted weights were generated by reversing these binary weights at randomly selected $10\%$ of edges.
The ground-truth smooth signal $\trueSignal$ was constructed from the low-frequency components of the true graph: the eigenvectors corresponding to the smallest eigenvalues of the graph Laplacian matrix, which give the smallest values of the GLR penalty (\ref{eq:glr}).
Specifically, $30\%$ of the low-frequency components were combined with the coefficients drawn from the Gaussian distribution with the standard deviation $1$, and then shifted and scaled to the range $[0, 1]$.
The box constraint was accordingly set with $\underline{\mu} = 0$ and $\overline{\mu} = 1$. 
The observation matrix $\PHI$ was set to a mask on randomly selected $20\%$ of elements for inpainting and an identity matrix for denoising, and the Gaussian noise with the standard deviation $0.05$ was added to the non-masked elements.
The fidelity constraint was set with $r = 0.9 \cdot 0.05 \sqrt{m}$, with $m$ as the number of non-masked elements.

We found the best performance of the proposed regularizations over the balancing parameters $\alpha \in [0, 1]$ through a grid search with the stride $0.005$ excluding $0$ and $1$.
The performances of GLR and GTV were observed with the same formulation and algorithm as the proposed methods, where $\p$ and the balancing parameter were no longer needed.
All optimizations were performed with the initial variables set to $\zeroVector$ and the stopping criterion set to $\lVert \u^{(n+1)} - \u^{(n)} \rVert_\text{P-PDS} < 10^{-9}$, where $\lVert \cdot \rVert_\text{P-PDS}$ is the norm equipped in the primal-dual space of P-PDS \cite{chambolle2011first}.
We evaluated the recovery performance by the (normalized) root mean squared error (RMSE) between the true signal $\trueSignal$ and the recovered signal $\u$: $\text{RMSE} = \ltwo{\u - \trueSignal} / \ltwo{\trueSignal}$, with the lower values indicating better performance.